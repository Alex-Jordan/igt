
\section{More Two-Person Zero-Sum Games: Dominated Strategies}

%\markright{More Two Player Zero-Sum Games}

\vspace{.2in}


Recall that in a zero-sum game, we know that one player's win is the other player's loss. Furthermore, we know we can rewrite any zero-sum game so that the player's payoffs are in the form $(a, -a)$. Note, this works even if $a$ is negative; in which case, $-a$ is positive.

{\bf Example 1.} Consider the following zero-sum game.

\hspace{1.2in}Player 2

\begin{tabular}{l|c|c|}\cline{2-3}
Player 1&(1, -1)&(0, 0)\\ \cline{2-3}
&(-1, 1)&(-2, 2)\\ \cline{2-3}

\end{tabular}
\medskip

If we know we are playing a zero-sum game, then the use of ordered pair seems somewhat redundant: If Player 1 wins 1, then we know that Player 2 must lose 1 (win $-1$). Thus, if we KNOW we are playing a zero-sum game, we can simplify our notation by just using Player 1's payoffs. For example, the above matrix can be simplified to the following matrix. 

\hspace{.8in}Player 2

\begin{tabular}{l|c|c|}\cline{2-3}
Player 1&1&0\\ \cline{2-3}
&-1&-2\\ \cline{2-3}

\end{tabular}
\medskip

When simplifying, keep a few things in mind:
\begin{enumerate}
\item You MUST know that the game is zero-sum.
\item If it is not otherwise specified, the payoffs represent Player 1's payoffs.
\item You can always give a similar matrix representing Player 2's payoffs. However, due to (2), you should indicate that the matrix is for Player 2. For example, Player 2's payoff matrix would be given by

\hspace{.8in}Player 2

\begin{tabular}{l|c|c|}\cline{2-3}
Player 1&-1&0\\ \cline{2-3}
&1&2\\ \cline{2-3}

\end{tabular}\ .

\medskip
\item Both players can make strategy decisions by considering only Player 1's payoff matrix. (Why?) Just to test this out, by looking only at the matrix 

\hspace{.8in}Player 2

\begin{tabular}{l|c|c|}\cline{2-3}
Player 1&1&0\\ \cline{2-3}
&-1&-2\\ \cline{2-3}

\end{tabular}
\medskip

 determine which strategy each player should choose.

\end{enumerate}

In this last example, it should be clear that Player 1 is looking for rows which give her the largest payoff-- this is nothing new. However, Player 2 is now looking for columns which give Player 1 the SMALLEST payoff. (Why?) 

Now that we have simplified our notation for zero-sum games, let's try to find a way to determine the best strategy for each player.

\break

{\bf Example 2.} Consider the following zero-sum game.

\hspace{1in}Player 2

\begin{tabular}{l|c|c|c|}\cline{2-4}
Player 1&1&0&2\\ \cline{2-4}
&-1&-2&2\\ \cline{2-4}

\end{tabular}
\medskip

Determine which row Player 1 should choose. Is there any situation in which Player 1 would choose the other row? 

\bigskip

{\bf Example 3.} Consider the following zero-sum game.

\hspace{1in}Player 2

\begin{tabular}{l|c|c|c|}\cline{2-4}
Player 1&1&0&2\\ \cline{2-4}
&-1&-2&3\\ \cline{2-4}

\end{tabular}
\medskip

Determine which row Player 1 should choose. Is there any situation in which Player 1 would choose the other row? 

\bigskip

In Example 2, no matter what Player 2 does, Player 1 would always choose Row 1, since every payoff in Row 1 is greater than or equal to the corresponding payoff in Row 2 ($1\ge -1$, $0\ge -2$, $2\ge 2$). In Example 3, this is not the case: If Player 2 were to choose Column 3, then Player 1 would prefer Row 2. In Example 2 we would say that Row 1 {\it dominates} Row 2.


{\bf Definition.} A strategy $X$ {\it dominates} a strategy $Y$ if every entry for $X$ is greater than or equal to the corresponding entry for $Y$. In this case, we say $Y$ is {\it dominated by} $X$.

In mathematical notation: The $i^{\rm th}$ row dominates the  $j^{\rm th}$ row if $a_{ik}\ge a_{jk}$ for all $k$, and $a_{ik}> a_{jk}$ for at least one $k$. If $X$ dominates $Y$, we can write $X\succ Y$.

This definition can also be used for Player 2: we consider columns instead of rows. If we are looking at Player 1's payoffs, then Player 2 prefers smaller payoffs. Thus one column $X$ dominates another column $Y$ if all the entries in $X$ are smaller than or equal to the corresponding entries in $Y$.  

Here is the great thing: we can always eliminate dominated strategies! (Why?)
Thus, in Example 2, we can eliminate Row 2.

\hspace{1in}Player 2

\begin{tabular}{l|c|c|c|}\cline{2-4}
Player 1&1&0&2\\ \cline{2-4}
&-1&-2&2\\ \cline{2-4}

\end{tabular}
\begin{picture}(0,0)
\put(-70,-5){\line(1,0){72}}
\end{picture}

\medskip
 Now it is easy to see what Player 2 should do.
 
 In Example 3, we cannot eliminate Row 2 since it is not dominated by Row 1. However, it should be clear that Column 2 dominates Column 3 (remember, Player 2 prefers SMALLER columns). Thus we can eliminate Column 3.
 
 \hspace{1in}Player 2

\begin{tabular}{l|c|c|c|}\cline{2-4}
Player 1&1&0&2\\ \cline{2-4}
&-1&-2&3\\ \cline{2-4}

\end{tabular}
\begin{picture}(0,0)
\put(-12,-12){\line(0,1){32}}
\end{picture}

\medskip
 AFTER eliminating Column 3, Row 1 dominates Row 2:
 
 \hspace{1in}Player 2

\begin{tabular}{l|c|c|c|}\cline{2-4}
Player 1&1&0&2\\ \cline{2-4}
&-1&-2&3\\ \cline{2-4}

\end{tabular}
\begin{picture}(0,0)
\put(-12,-12){\line(0,1){32}}
\put(-70,-5){\line(1,0){50}}
\end{picture}
\medskip

Again, now it is easy to determine what each player should do.

{\bf Exercise.} Check that the strategy pairs we determined in Examples 2 and 3 are, in fact, equilibrium pairs.

{\bf Exercise.} Use the idea of eliminating dominated strategies on the Arnold/ Bainbridge examples from the previous section. Do you get the same strategy pairs?


 