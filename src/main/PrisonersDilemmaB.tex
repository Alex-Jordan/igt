
\section{Prisoner's Dilemma and Chicken}

%\markright{Prisoner's Dilemma}

%\vspace{.2in}
%Do all of your work on your own paper. Give complete answers (complete sentences!).

\vspace{.1in}
In this section we look at two classic non-zero-sum games. 

The first game is called {\bf Prisoner's Dilemma}. Two partners in crime are arrested for  burglary and sent to separate rooms. They are each offered a deal: if they confess and rat on their partner, they will receive a reduced sentence. So if one confesses and the other doesn't, the confessor only gets 3 months in prison, while the partner serves 10 years. If both confess, then they each get 8 years. However, if neither confess, there isn't enough evidence, and each gets just one year. We can represent the situation with the following matrix.

\hspace{3in}Prisoner 2

\begin{center}
\begin{tabular}{l|r|c|c|}\cline{2-4}
&&\textbf{Confess}&\textbf{Don't confess}\\ \cline{2-4}
Prisoner 1&\textbf{Confess} &(8, 8)&(0.25, 10)\\ \cline{2-4}
&\textbf{Don't confess} &(10, 0.25)&(1, 1)\\ \cline{2-4}
\end{tabular}
\end{center}
\vspace{.1in}

\begin{enumerate}
\item Does the above matrix have any dominated strategies for Player 1? Does it have any dominated strategies for Player 2? Keep in mind that a prisoner prefers smaller numbers since prison time is bad.

\item Suppose you are Prisoner 1. What should you do? Why? Suppose you are Prisoner 2. What should you do? Why? Does your choice of strategies result in an equilibrium pair?

\item Looking at the game as an outsider, what strategy pair is the best option for both prisoners. 

\item Now suppose both prisoners are perfectly rational, so that any decision Prisoner 1 makes would also be the decision Prisoner 2 makes. Further, suppose both prisoners know that their opponent is perfectly rational. What should each prisoner do?

\item Suppose Prisoner 2 is crazy and is likely to confess with 50/50 chance. What should Prisoner 1 do? Does it change if he confesses with a 75\% chance? What if he confesses with a 25\% chance.

\item Suppose the prisoners are able to communicate about their strategy. How might this affect what they choose to do?

\item Explain why this is a ``dilemma" for the prisoners. Is it likely they will chose a strategy which leads to the best outcome for both? You might want to consider whether there are dominated strategies.

%\vspace{.1in}
\end{enumerate}

\break
The second game is called {\bf Chicken}. Two drivers drive towards each other. If one driver swerves, he is considered a ``chicken." if a driver doesn't swerve (drives straight), he is considered the winner. Of course if neither swerves, they crash and neither wins. A possible payoff matrix for this game is

\hspace{3in}Driver 2

\begin{center}
\begin{tabular}{l|r|c|c|}\cline{2-4}
&&\textbf{Swerve}&\textbf{Straight}\\ \cline{2-4}
Driver 1&\textbf{Swerve} &(0, 0)&(-1, 10)\\ \cline{2-4}
&\textbf{Straight} &(10, -1)&(-100, -100)\\ \cline{2-4}
\end{tabular}
\end{center}
\vspace{.1in}

\begin{enumerate}
\setcounter{enumi}{7}

\item Does this game have any dominated strategies?
%\vspace{.1 in}

\item What strategy results in the best outcome for Player 1? What strategy results in the best outcome for Player 2? What happens if they both choose that strategy?
%\vspace{.1 in}

\item Consider the strategy pair with outcome (-1, 10). Does Player 1 have any regrets about his choice? What about Player 2? Is this an equilibrium pair? Are there any other points in which neither player would regret his choice?
%\vspace{.1in}

\item Can you determine what each player should do in this game? If so, does this result in an equilibrium pair?
%\vspace{.1 in}

\item Now suppose both players in the game of chicken are perfectly rational, so that any decision Player 1 makes would also be the decision Player 2 makes. Further, suppose both players know that their opponent is perfectly rational. What should each player do?%\vspace{.1 in}

\item Suppose Player 2 is a remote control dummy and will swerve or drive straight with a 50/50 chance. What should Player 1 do? Does it change if he swerves with 75\% chance?

\item Can it benefit players in the game of chicken to communicate about their strategy? Explain.


\item Compare Prisoner's Dilemma and Chicken. Are there dominated strategies in both games? Are there equilibrium pairs? Are players likely to reach the optimal payoff for one player, both players, or neither player? Does a player's choice depend on what he knows about his opponent (perfectly rational or perfectly random)?

\end{enumerate}



 