

\section{Summary of Strategies for Zero-Sum Games}

%\markright{Equilibrium Point Experiment}

%\vspace{.2in}
%Do all of your work on your own paper. Give complete answers (complete sentences!).

\vspace{.1in}
In this section, we will try to understand where we are with solving two-player zero-sum games. %In general, we call the pair of strategies played an {\it equilibrium pair}, while we call the specific payoff vector associated with an equilibrium pair an {\it equilibrium point}.

\begin{enumerate}
\item Write down a random payoff (zero-sum) matrix with two strategy choices for each player.
\item Write down a random payoff (zero-sum) matrix with three strategy choices for each player.
\item Write down a random payoff (zero-sum) matrix with four strategy choices for each player.
\item Exchange your list of matrices with another student in the class. For each matrix you have been given


\begin{enumerate}
\item try to determine any dominated strategies, if they exist.
\item try to determine any equilibrium points, if they exist.
\item determine the maximin and minimax strategies for Player 1 and Player 2, respectively. Can you always find these?
 
\end{enumerate}
%\vspace{.1in}

\item Now combine all the examples of payoff matrices in a group of 3 or 4 students. Make a list of the examples with equilibrium points and a list of examples without equilibrium points. If you have only one list, try creating examples for the other list. Based on your lists, do you think random payoff matrices are likely to have equilibrium points? 

\vspace{.1in}

\end{enumerate}

Now we want to use the lists of matrices as experimental examples to try to answer some of the remaining questions we have about finding rational solutions for games and equilibrium points. If you don't feel you have enough examples, you are welcome to create more or gather more from your classmates.

\begin{enumerate}
\setcounter{enumi}{5}

\item If a matrix has an equilibrium point, can a player ever do better to {\it not} play an equilibrium strategy? Explain.%\vspace{.1in}

\item If a matrix has an equilibrium point, does the maximin/minimax strategy always find it? Explain.

\item If a matrix doesn't have an equilibrium point, should  player always play the maximin/minimax strategy? Explain.
%\vspace{.1in}

\item If a matrix doesn't have an equilibrium point is there an ideal strategy for each player? Explain.
%\vspace{.1in}

\item Write a brief summary of the connections you have observed between finding a rational solution for a game and equilibrium points. 

\end{enumerate}


%\end{document}


 